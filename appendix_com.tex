\section{Implementierung des DSXCS Algorithmus}

\newlisting{Erstes Kernst�ck des verz�gerten SXCS Algorithmus DSXCS (\emph{collectReward()}, Bewertung der \emph{action set} Listen)}{collect_reward_dsxcs:pro}
/**
 * Diese Funktion verarbeitet den �bergebenen base reward Wert und 
 * gibt ihn an die zugeh�rigen action set Listen weiter. Es wird
 * kein maxPrediction Wert berechnet oder weitergegeben.
 *
 * @param start_index Index in der historicActionSet Liste der als 
 *          erstes aktualisiert werden soll.
 * @param action_set_size Anzahl der action set Listen, die aktualisiert 
 *          werden sollen.
 * @param reward Wahr wenn der Zielobjekt in Sicht oder keine Agenten 
 *          in Sicht waren.
 * @param factor Der Aktualisierungsfaktor
 * @param is_event Wahr wenn diese Funktion wegen eines positiven oder 
 *          negativen Ereignisses aufgerufen wurde.
 */
  public void collectReward(final int start_index, 
                final int action_set_size, final boolean reward, 
                final double factor, final boolean is_event) {

    double corrected_reward = reward ? 1.0 : 0.0;

  /**
   * Aktualisiere eine ganze Anzahl von Eintr�gen in der 
   * historicActionSet Liste.
   */
     for(int i = 0; i < action_set_size; i++) {

    /**
     * Benutze aufsteigenden bzw. absteigenden reward Wert bei 
     * einem positiven bzw. negativen Ereignis.
     */
       if(is_event) {
         corrected_reward = reward ? 
           calculateReward(i, action_set_size) : 
           calculateReward(action_set_size - i, action_set_size);
       } 
    /**
     * F�ge den ermittelten reward Wert zur historicActionSet Liste.
     */
       historicActionSet.get(start_index - i).
         addReward(corrected_reward, factor);
    }
  }
\end{lstlisting}


\newlisting{Zweites Kernst�ck des verz�gerten SXCS Algorithmus DSXCS (\emph{calculateNextMove()}, Auswahl der n�chsten Aktion und Ermittlung der zugeh�rigen \emph{action set} Liste)}{dsxcs_calc_move:pro}

/**
 * Der erste Teil der Funktion ist identisch mit der calculateNextMove()
 * Funktion der SXCS Variante ohne Kommunikation. Der Zusatz ist, dass beim 
 * �berlauf die in der historicActionSet Liste gespeicherte reward Werte
 * verarbeitet werden.
 *
 * @param gaTimestep Der aktuelle Zeitschritt
 */

  public void calculateNextMove(final long gaTimestep) {

  // ... 

  /**
   * historicActionSet Liste voll? Dann verarbeite den dortigen reward Wert.
   */
     if (historicActionSet.size() > Configuration.getMaxStackSize()) {
       historicActionSet.pop().processReward();
    }
  }
\end{lstlisting}


\newlisting{Drittes Kernst�ck des verz�gerten SXCS Algorithmus DSXCS  (Verarbeitung des jeweiligen reward Werts, \emph{processReward()})}{process_reward_dsxcs1:pro}
/**
 * Zentrale Routine der historicActionSet Liste zur Verarbeitung aller 
 * eingegangenen reward Werte bis zu diesem Punkt.
 */

  public void processReward() {

    for(RewardHelper r : reward) {
    /**
     * Aktualisiere den Eintrag mit den entsprechenden Werten.
     */
      actionClassifierSet.updateReward(r.reward, 0.0, r.factor);
    }
  }
\end{lstlisting}


\newlisting{Verbesserte Variante des dritten Kernst�ck des verz�gerten SXCS Algorithmus DSXCS (Verarbeitung des jeweiligen reward Werts, \emph{processReward()})}{process_reward_dsxcs2:pro}
/**
 * Verbesserte zentrale Routine der historicActionSet Liste zur 
 * Verarbeitung aller eingegangenen reward Wert bis zu diesem Punkt.
 */

  public void processReward() {

    double max_value = 0.0;
    double max_reward = 0.0;
  /**
   * Finde das gr��te reward / factor Paar.
   */
    for(RewardHelper r : reward) {
      
      if(r.reward * r.factor > max_value) {
        max_value = r.reward * r.factor;
        max_reward = r.reward;
      }
    }
    /**
     * Aktualisiere den Eintrag mit dem ermittelten Werten.
     */
    actionClassifierSet.updateReward(max_reward, 0.0, 1.0);
  }
\end{lstlisting}

\clearpage




\section{Implementierung des egoistischen \emph{reward}}

\newlisting{"'Egoistische Relation"', Algorithmus zur Bestimmung des Kommunikationsfaktors basierend auf dem erwarteten Verhalten des Agenten gegen�ber anderen Agenten}{egoistic_relationship:pro}
/**
 * �hnlichkeit dieser classifier set Liste zu der �bergebenen Liste im
 * Hinblick auf die Wahrscheinlichkeit, auf andere Agenten zuzugehen
 * @param other Die andere Liste mit der verglichen werden soll
 * @return Grad der �hnlichkeit bzw. der Kommunikationsfaktor (0,0 - 1,0)
 */
  public double checkEgoisticDegreeOfRelationship(
                    final MainClassifierSet other) {
    double ego_factor = getEgoisticFactor() - other.getEgoisticFactor();
    if(ego_factor == 0.0) {
      return 0.0;
    } else {
      return 1.0 - ego_factor * ego_factor;
    }
  }

  public double getEgoisticFactor() throws Exception {
    double factor = 0.0;
    double pred_sum = 0.0;
    for(Classifier c : getClassifiers()) {
      if(!c.isPossibleSubsumer()) {
        continue;
      }
      factor += c.getEgoFactor();
      pred_sum += c.getFitness() * c.getPrediction();
    }
    if(pred_sum > 0.0) {
      factor /= pred_sum;
    } else {
      factor = 0.0;
    }
    return factor;
  }
\end{lstlisting}




