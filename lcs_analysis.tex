\chapter{Analyse SXCS}\label{lcs_analysis:cha}







\begin{table}[ht]
\caption{Vergleich ``Intelligent (Open)'' und ``Intelligent (Hide)'' (8 Agenten, S�ulenszenario)}
\centering
\begin{tabular}{c c c}
\hline\hline
Algorithmus & Abdeckung & Qualit�t \\ [1ex]
\hline
``Intelligent (Open)'' \\ [1ex]
\hline
Zuf�llige Bewegung     & 72.55\% & 11.58\% \\
XCS                    & 71.35\% & 13.98\% \\
SXCS                   & 72.10\% & 13.50\% \\ [1ex]
\hline
``Intelligent (Hide)'' \\ [1ex]
\hline
Zuf�llige Bewegung     & 72.56\% & 11.78\% \\
XCS                    & 71.33\% & 14.27\% \\
SXCS                   & 72.05\% & 13.90\% \\ [1ex]
\hline
\end{tabular}
\label{table:intelligent_open_hide_pillar}
\end{table}


\begin{table}[ht]
\caption{Vergleich ``Intelligent (Open)'' und ``Intelligent (Hide)'' (8 Agenten, S�ulenszenario)}
\centering
\begin{tabular}{c c c}
\hline\hline
Algorithmus & Abdeckung & Qualit�t \\ [1ex]
\hline
``Intelligent (Open)'' \\ [1ex]
\hline
Zuf�llige Bewegung     & 72.55\% & 11.58\% \\
XCS                    & 71.35\% & 13.98\% \\
SXCS                   & 72.10\% & 13.50\% \\ [1ex]
\hline
``Intelligent (Hide)'' \\ [1ex]
\hline
Zuf�llige Bewegung     & 72.56\% & 11.78\% \\
XCS                    & 71.33\% & 14.27\% \\
SXCS                   & 72.05\% & 13.90\% \\ [1ex]
\hline
\end{tabular}
\label{table:intelligent_open_hide_pillar}
\end{table}


TODO auch sich langsam bewegende analysieren!
Und auch stehenbleibende :> z.B. im Raumszenario.

Geschwindigkeit 2 problematisch, Geschwindigkeit 1 ok

lcs l�nger laufen lassen!viele experimente

\section{Zusammenfassung der bisherigen Erkenntnisse}

Bez�glich der Tests konnten in den vorangegangenen Kapiteln bisher folgende Ergebnisse in Erfahrung gebracht werden:

\begin{itemize}
\item Algorithmen mit Ergebnissen die unter dem des zuf�lligen Algorithmus liegt, sind unbrauchbar und nicht vergleichbar. ``Verbesserungen'', die die Qualit�t des Algorithmus n�her an das Ergebnis des zuf�lligen Algorithmus bringen, sind in Wirklichkeit Ver�nderungen, die den Algorithmus eher zuf�llige Entscheidungen treffen lassen, und keine tats�chlichen Lernerfolge.
\item Szenarien
\item 

\end{itemize}

SXCS sehr gut bei NO DIRECTION CHANGE und speed 1!


nicht geschafft: Pillar, one direction change, speed 2, XCS ...besser... weil zuf�lliger


\section{Standard XCS Multistepverfahren}



\subsection{SXCS und Heuristiken}

erst multistep... mit random vergleichen

In allen Tests erreichten die Heuristiken deutlich bessere Ergebnisse. Diesen Nachteil hat sich LCS in diesen Szenarien durch deutlich �berlegene Flexibilit�t erkauft
Ein Gro�teil der eingehenden Informationen ist f�r die Auswertung nicht relevant und lokale Information ist zu ungenau.
Bei einer komplexeren Implementierung mit Distanzen

Insbesondere der Vergleich mit dem intelligenten Agenten, der anderen Agenten ausweicht, zeigt, dass die LCS Agenten unm�glich ein solches globales Ziel erreichen k�nnen, es ist also kein emergentes Verhalten zu beobachten. Dies ist dadurch zu begr�nden, dass bei der Berechnung des Rewards keine Information au�er der eigenen, lokalen Information 

der Abstand zu anderen Agenten nicht Teil der Berechnung des Rewards ist, noch gibt keine eingebaute Heuristik. Man k�nnte zwar 


TODO statistical value:Error in predictions!

\subsection{Vergleich Multistep / LCS}

Szenarien, Parameter.

\subsection{Test der verschiedenen Exploration-Modi}


Prediction Error sehr hoch, da dynamisches 
