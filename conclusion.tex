\chapter{Zusammenfassung, Ergebnis und Ausblick}
\label{conclusion:cha}


\section{Zusammenfassung}
Zu Beginn wurde auf die Szenariodefinition und die F�higkeiten der Agenten eingegangen. Anhand von Beispielen heuristischer Agenten wurden einige Grundeigenschaften der pr�sentierten Szenarien als Vorbereitung f�r die Analyse der Learning Classifier Systeme bestimmt. Nach der Einf�hrung in LCS, der Beschreibung des Standardverfahren XCS und der angepassten Implementierung f�r �berwachungsszenarios konnten dann umfangreiche Tests ausgef�hrt werden. 


von der M�glichkeit zur Kommunikation eine angepasste Implementierung f�r verz�gerten Reward definiert auf Basis dessen dann mehrere Varianten f�r die Weitergabe des Rewards vorgestellt, analysiert und verglichen wurden.

\section{Ergebnis}
Das wesentliche Ergebnis ist, dass die Implementierung des XCS auf  �berwachungsszenarios ausgeweitet werden kann ohne wesentliche Ver�nderungen am Algorithmus vorzunehmen. W�hrend sich die Qualit�t der resultierenden Agenten im Allgemeinen �ber dem zuf�lligen Agenten befindet, ist die Effizienz der Implementierung, im Vergleich zu einfachen Heuristiken, sehr gering. Mit der verwendeten Implementierung hat XCS Probleme, eine optimale Regelmenge zu finden bzw. zu halten. Eine Regel wie z.B. ``laufe auf das Ziel zu, wenn es in Sicht ist'', ist als Heuristik sehr erfolgreich, bei dauerhafter �berwachung ohne Kommunikation l�uft es aber eher auf ein Verfolgungsszenario hinaus. Aufgrund andauerndem Lernens TODO

Die alleinige Anpassung des XCS Multistepverfahrens, dass ein neues Problem gestartet wird, wann immer sich das Ziel in �berwachungsreichweite befand f�hrte nicht zum Erfolg, die Ergebnisse waren nicht besser als ein sich zuf�llig bewegender Agent.\\


Erst durch Verkn�pfung des Rewards mit dem zeitlichen Abstand zu einer �nderung des Zustands f�hrte zu deutlich besseren Ergebnissen.\\ TODO
Desweiteren wurde untersucht, inwiefern sich der Austausch an minimaler Information unter den Agenten, ohne zentrale Steuerung oder globalem Regeltausch, auf die Qualit�t auswirkt. Zwar gab es vereinzelt positive Effekte, diese waren jedoch auf andere Faktoren zur�ckzuf�hren.

\section{Ausblick}
Ein 


Weitere Untersuchungen sind n�tig um zu bestimmen, inwiefern Kommunikation, beispielsweise mit einer gr��eren Zahl an besseren Sensoren, zu einem besseren Ergebnis f�hren kann. TODO\\
Vom theoretischen Standpunkt ist noch zu kl�ren, warum genau der zeitliche Abstand zum Erfolg gef�hrt hat und wo die Grenzen hierf�r liegen. 

Erschwerung, mehr Kollaboration
TODO aus verschiedenen Richtungen betrachten? Mehrere Agenten notwendig?

Probiert, aber verworfen:
Rotation
Numerosity



\chapter{Verwendete Hilfsmittel und Software}

Zu Beginn stellte sich die Frage, welche Software zu benutzen ist, da es sich um ein recht komplexe Problemstellung handelt. Begonnen habe ich mit der YCS Implementierung \cite{Bull}. Sie ist in der Literatur wenig vertreten, die Implementierung bot aber einen guten Einstieg in das Thema, da sie sich auf das Wesentliche eines LCS beschr�nkte und keine Optimierungen enthielt.\\

Der n�chste Schritt war zu entscheiden, auf welchem System die Agenten simuliert werden sollen. Implementierungen wie 

Unter einer Reihe von vorhandenen Implementierungen entschied ich mich f�r eine eigene Implementation. 

Wesentlicher Grund war die Unerfahrenheit mit den L�sungen (und der damit verbundenen Einarbeitungszeit) wie auch �berlegungen bzgl. der Geschwindigkeit, dem Speicherverbrauch und der Kompatibilit�t. TODO

Das Programm und die zugeh�rige Oberfl�che zum Erstellen von Test-Jobs wurden in Netbeans 6.5 programmiert.

Grafiken wurden mittels GnuPlot erstellt.

Grafiken der Grid-Konfiguration wurden im Programm mittels GifEncode TODO erste
 * @version 0.90 beta (15-Jul-2000)
 * @author J. M. G. Elliott (tep@jmge.net)

Wesentlicher Bestandteil der Konfigurationsoberfl�che war auch eine Automatisierung der Erstellung von Konfigurationsdateien, Batchdateien (f�r ein Einzelsystem und f�r JoSchKA) zum Testen einer ganzen Reihe von Szenarien und auch GnuPlot Skripts.

Speicherverbrauch

Speicherung der Agentenpositionen und des Grids verbrauchen fast keinen Speicher TODO
Wesentlicher Faktor waren die LCS Systeme mit ihren ClassifierSets TODO

OpenOffice

LEd Latex




\section{Beschreibung des Konfigurationsprogramms}

\begin{figure}[htbp]
\centerline{	
%\includegraphics{agent_configuration.eps}
}
\caption[Screenshot des Konfigurationsprogramms] {Screenshot des Konfigurationsprogramms}
\label{agent_configuration:fig}
\end{figure}



