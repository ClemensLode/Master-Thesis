\chapter{Einf�hrung}\label{introduction:cha}

Ein aktuelles Forschungsgebiet aus dem Bereich der \emph{learning classifier systems} (LCS) stellen die sogenannten \emph{accuracy based} LCS (XCS) dar. In der Basis entspricht XCS einem LCS, d.h. eine Reihe von Regeln, bestehend jeweils aus einer Kondition und einer Aktion, werden mittels \emph{reinforcement learning} schrittweise bewertet und an eine Umwelt angepasst. Die Frage nach dem Zeitpunkt der Bewertung teilt die verwendeten Algorithmen bei XCS in \emph{single step} und \emph{multi step} Verfahren ein. Hauptaugenmerk dieser Arbeit soll das \emph{multi step} Verfahren sein, bei dem die Bewertung (der \emph{reward} der Regeln erst nach einigen Schritten verf�gbar ist und an zur�ckliegende Regeln sukzessive weitergeleitet wird um m�glichst alle beteiligten Regeln an dem \emph{reward} zu beteiligen.\\

Bisherige Anwendungen haben sich haupts�chlich auf statische Szenarien mit nur einem XCS oder mit mehreren Agenten mit globaler Organisation und Kommunikation beschr�nkt. Diese Arbeit hat sich auf die Problemstellung konzentriert, wie man XCS modifizieren sollte, damit es ein dynamisches �berwachungsszenario, mit sich bewegendem Zielobjekt und mehreren Agenten, im Vergleich zu zuf�lliger Bewegung m�glichst gut bestehen.\\

Die Zahl der m�glichen Anpassungen, insbesondere was das Szenario, die XCS Parameter und Anpassungen an die XCS Implementierung betrifft, sind un�berschaubar gro� und bed�rfen in erster Linie einer theoretischen Basis, welche in diesem Bereich noch nicht weit fortgeschritten ist. Ziel dieser Arbeit soll es deshalb sein, zu untersuchen, welche Anpassungen speziell f�r das �berwachungsszenario erfolgsversprechend sind.\\

*Empirisch

Im Wesentlichen wurde hierzu in zwei Schritten vorgegangen, die auch in der Struktur der Arbeit wiedergespiegelt sind, um eine logische Kette aufzubauen. Der erste Schritt soll sich alleine um die Beschreibung des Problems und des Szenarios drehen. 
Literatur Szenarien, Woods Maze etc.

TODO 




Neben der Anpassung der Implementation, damit XCS f�r eine solche Problemstellung anwendbar ist, wurden weitere Modifikationen durchgef�hrt, die in einigen F�llen zu deutlich besseren Ergebnissen als die der Standardimplementation f�hrten.\\
Au�erdem wurde untersucht, wie eine einfache Kommunikation ohne globale Steuereinheit stattfinden kann, um das Ergebnis weiter zu verbessern. Im Wesentlichen war dazu eine weitere Anpassung von XCS vonn�ten, so dass die Implementierung auch mit (durch die Kommunikation) zeitverz�gerten und externen \emph{rewards} arbeiten konnte. Wesentliche Schlu�folgerung ist, dass sich unterschiedliche Szenarien unterschiedlich gut f�r Kommunikation eignen, dass Kommunikation M�glichkeiten zur Anpassung bietet um mit einer variablen, unbekannten Feldgr��e besser zurecht zu kommen und, dass es Szenarien gibt, in denen Kommunikation signifikante Vorteile erbringt.\\
Erfolgversprechende Ansatzpunkte f�r weitere Forschung gibt es im Bereich der mathematischen Begr�ndung, warum die Implementierung Vorteile erbringt, im Ausbau der Untersuchung von Kommunikation zwischen den Agenten in Verbindung mit XCS und in der Anwendung der gefundenen Ergebnisse in anderen Problemstellungen �hnlicher Natur.
