\chapter{Einf�hrung}
\label{introduction:cha}

XCS ist ein aktuelles Forschungsgebiet aus dem Bereich der Learning Classifier Systeme. Bisherige Anwendungen haben sich haupts�chlich auf statische Szenarien mit nur einem Agenten oder mit mehreren Agenten mit globaler Organisation und Kommunikation beschr�nkt. Statisch in dem Sinne, dass in jedem untersuchten Problem das Ziel jeweils eine feste Position einnahm, das Hauptaugenmerk also darauf gerichtet war, anhand der sich auf dem Feld befindlichen Hindernisse einen m�glichst kurzen Weg zum Ziel zu finden. Dieses Verfahren nennt sich ``Multistepverfahren'' bei dem der entg�ltige Reward an andere Schritte zur�ckgegeben wird.\\
Diese Arbeit wird sich um die Problemstellung k�mmern, wie man XCS modifizieren sollte, damit es sich m�glichst gut in einem dynamischen Szenario mit mehreren Agenten und einem sich bewegenden Ziel zurechtfinden kann. Durch die Modifikation konnten deutlich bessere Ergebnisse erzielt werden wie im Standard-Multistepverfahren.\\
Au�erdem wurde untersucht, wie eine Kommunikation auf minimaler Basis ohne globale Steuereinheit stattfinden kann um das Ergebnis zu verbessern. Wesentliche Schlu�folgerung ist, dass sich unterschiedliche Szenarien unterschiedlich gut f�r Kommunikation eignen, dass Kommunikation M�glichkeiten zur Anpassung bietet um mit einer variablen, unbekannten Feldgr��e zurecht zu kommen und, dass es Szenarien gibt, in denen Kommunikation signifikante Vorteile erbringt.\\

Da die Menge und Qualit�t der verwendeten (simulierten) Sensoren auf niedrigstem Niveau gehalten wurde gibt es hier gute Ansatzpunkte f�r weitere Forschung in diesem Bereich.



TODO Einleitung aus LCS raus
