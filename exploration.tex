\chapter{Die \emph{exploration} und \emph{exploitation} Phasen}\label{cha:exploration}

Bei der Auswahl des \emph{classifiers} in 

\section{Auswahlmethode}

\subsection{Die \emph{roulette wheel} Auswahlmethode}

\subsection{Die \emph{tournament selection} Auswahlmethode}


\section{Die \emph{exploration} Phase}

F�r die \emph{exploration} Phase soll es bei der Implementation zwei M�glichkeiten geben:
\begin{enumerate}
\item ``Zuf�llige Auswahl'': Zuf�llige Auswahl eines \emph{classifier}, unabh�ngig von \emph{fitness} oder \emph{prediction}
\item ``Roulette wheel Auswahl'': Zuf�llige Auswahl eines \emph{classifier}, mit Wahrscheinlichkeit abh�ngig von dessen \emph{fitness} * \emph{prediction} Produkts
\end{enumerate}

Bei einem dynamischen �berwachungsszenario ist es im Vergleich zu standardm��igen statischen Szenarien weder n�tig noch hilfreich ``zuf�llige Auswahl'' zu nutzen, da die Idee hierf�r in einem statischen Szenario ist, dass man vermeiden m�chte, dass das LCS immer wieder die selben Entscheidungen trifft und somit immer wieder den selben Umweltreizen ausgesetzt ist, was wiederum zu immer wieder gleichen Entscheidungen f�hrt usw.\\
Bei einem dynamischen Szenario ergibt sich das Problem aber nicht, andere Agenten und das Ziel sind in stetiger Bewegung, der eigene Startpunkt ist nicht fixiert und das Problem wird bei Erreichen des Ziels nicht neugestartet. Es ist zu erwarten, dass ``roulette wheel Auswahl'' ausreicht oder auf \emph{exploration} wom�glich v�llig verzichtet werden kann.\\


\section{Die \emph{exploit} Phase}

F�r die \emph{exploit} Phase ergibt sich neben der Auswahl des besten (d.h. desjenigen mit h�chstem Produkt aus \emph{fitness} * \emph{prediction}) \emph{classifiers} eine zweite M�glichkeit, die in \cite{Butz2003} (``tournament selection'') diskutiert wurde. Die Turnierauswahl soll sich hier aber darauf beschr�nken, dass die \emph{classifier} Liste sortiert und nacheinander mit der Wahrscheinlichkeit \(p\) ein \emph{classifier} gew�hlt wird (d.h. der erste mit \(p\), der zweite mit \((1.0-p)*p\), der dritte mit \((1.0-p)(1.0-p)*p\) usw.).

Faktor p ermitteln, 0.6 scheint gut zu sein

\begin{enumerate}
\item ``Beste Auswahl'': Wahl jeweils des \emph{classifiers} mit dem gr��ten Produkt aus \emph{fitness} und \emph{prediction}
\item ``Turnierauswahl'': Wahl des jeweils besten \emph{classifiers} mit Wahrscheinlichkeit \(p\), Wahl des zweitbesten mit Wahrscheinlichkeit \((1.0-p)*p\) usw.
\end{enumerate}

No exploration => viele ung�ltige Bewegungen, nicht ``wegkommen'' von Hindernis / stehenbleiben?

TODO SEHR WICHTIG BEI SICH WENIGBEWEGENDENZIELEN


\section{Wechsel zwischen Exploration und Exploitation}\label{exploreexploit:sec}

Die Wahl der Auswahlart f�r \emph{classifier} in Punkt (3) in Kapitel \ref{ablauf_lcs:sec} kann auf verschiedene Weise erfolgen. 


In der Standardimplementierung von XCS wird zwischen ``exploit'' und ``explore'' nach jedem Erreichen des Ziels entweder umgeschalten oder zuf�llig mit einer bestimmten Wahrscheinlichkeit eine Auswahlart ermittelt. Es werden also abwechselnd ganze Probleme im ``exploit'' und ``explore'' Modus berechnet. Dies erscheint sinnvoll f�r die erw�hnten Standardprobleme, da nach Erreichen des Ziels ein neues Problem gestartet wird und die Entscheidungen die w�hrend der L�sung eines Problems getroffen werden keine Auswirkungen auf die folgenden Probleme hat, die Probleme also nicht miteinander zusammenh�ngen.\\
Bei dem hier vorgestellten �berwachungsszenario kann nicht neugestartet werden, es gibt keine ``Trocken�bung'', die Qualit�t eines Algorithmus soll deshalb davon abh�ngen, wie gut sich der Algorithmus w�hrend der gesamten Berechnung, inklusive der Lernphasen, verh�lt. Es ist nicht m�glich bei diesem Szenario zwischen ``exploit'' und ``explore'' Phasen zu differenzieren. Desweiteren greift auch die Idee einer reinen ``explore'' Phase beim �berwachungsszenario nicht, da das Szenario nicht statisch, sondern dynamisch ist. Ein zuf�lliges Herumlaufen kann, im Vergleich zur gewichteten Auswahl der Aktionen, dazu f�hren, dass der Agent mit bestimmten Situationen mit deutlich niedrigerer Wahrscheinlichkeit konfrontiert wird, da der Agent sich in Hindernissen verf�ngt oder das Zielobjekt ihm andauernd ausweicht. Aus diesen Gr�nden erscheint es sinnvoll, weitere Formen des Wechsels zwischen diesen Phasen zu untersuchen:

\begin{enumerate}
\item Abwechselnd f�r jedes Problem entweder ``explore'' oder ``exploit''
\item Mit Wahrscheinlichkeit \(p_{\mathrm{explore}}\) ``explore'', sonst  ``exploit''
\item Wechsel zwischen ``explore'' und ``exploit'' bei �nderung des \emph{reward} Werts
\end{enumerate}



M�glichkeit (3.) und (4.) entspricht dem Fall in der Standardimplementierung von XCS. Dabei wird bei jedem Erreichen eines positiven Rewards zwischen ``explore'' und ``exploit'' hin und hergeschaltet, was in der Standardimplementierung dem Beginn eines neuen Problems entspricht.


TODO Umschalten bei reward, Code evtl.

TODOTESTS

TODO SWITCH EXPLORE/EXPLOIT + NEW LCS sehr gut



Es ist zu erwarten, dass sich die Fitness/Prediction werte vieler Aktionen kaum Unterscheiden (Weitergabe etc.) => Turnierselection bessere Unterscheidung zwischen guten und schlechten Classifiers
