\chapter{Erste Analyse der Agenten ohne LCS}

In diesem Kapitel sollen erste Analysen bez�glich der verwendeten Szenarien anhand des zuf�lligen Algorithmus, des Algorithmus mit einfacher Intelligenz (``Simple AI Agent'') und des Algorithmus mit komplexerer Regeln (``Intelligent AI agent'') angefertigt werden. Die Ergebnisse aus der Analyse werden eine Grundlage f�r die vergleichende Betrachtung der Agenten mit LCS Algorithmen dienen, insbesondere werden sie Anhaltspunkte daf�r geben, welche Szenarien welche Eigenschaften der Algorithmen testen.

\section{``Total Random'' Zielobjekt}

In allen Szenarien mit dieser Form der Bewegung des Zielobjekts kommt es nur darauf an, dass die Agenten einen m�glichst gro�en Bereich des Torus abdecken.  Wie die folgenden Abschnitte zeigen werden, unterscheidet sich der ``Simple AI Agent'' nicht wesentlich vom zuf�lligen Agenten. Der intelligente Agent erreicht hier dagegen in den meisten F�llen deutlich bessere Ergebnisse.

\subsection{Ohne Hindernisse}

Ohne Hindernisse gibt sich ein klares Bild. Das Ergebnis der einfachen KI ist etwas schlechter als der des zuf�lligen Agenten, da sich immer wenn mehrere Agenten das Zielobjekt in der selben Richtung in Sichtweite haben, sich mehrere Agenten in die selbe Richtung bewegen. Dies beeintr�chtigt die zuf�llige Verteilung der Agenten auf dem Spielfeld und f�hrt somit auch zu einer niedrigeren Abdeckung des Torus.\\
Der intelligente Agent liegt hier sehr deutlich vorne, ein m�glichst weitr�umiges Verteilen auf dem Torus f�hrt zum Erfolg, denn genau das wird mit dem v�llig zuf�llig springenden Agenten getestet.

\begin{table}[ht]
\caption{``Total Random'' ohne Hindernisse}
\centering
\begin{tabular}{c c c c}
\hline\hline
Algorithmus & Qualit�t & Abdeckung & �berhang \\ [0.5ex]
%heading
\hline
Zuf�llige Bewegung & 75.88\% & 76.09\% & 59.54\% \\ % inserting body of the table
Einfache Heuristik & 75.80\% & 75.09\% & 60.10\% \\
Intelligente Heuristik & 87.53\% & 87.26\% & 53.10\% \\ [1ex] % [1ex] adds vertical space
\hline %inserts single line
\end{tabular}
\label{table:nonlin} % is used to refer this table in the text
\end{table}


\subsection{Zuf�llig verteilte Hindernisse}

Hier ergeben sich bei allen Einstellungen des ``Verkn�pfungsfaktors'' und ''Obstacle Factors'' ebenfalls ein klares Bild, der intelligente Agent liegt wieder vorne, dann kommt allerdings schon der einfache Agent mit bis zu 10\% zum zuf�lligen Agenten. Der wesentliche zweite Faktor ist hier, dass der einfache Agent, wenn er das Zielobjekt in Sicht hat, davon ausgehen kann, dass sich in dieser Richtung wahrscheinlich kein Hindernis befindet, w�hrend der zuf�llige Agent Hindernisse �berhaupt nicht beachtet, somit �fters gegen ein Hindernis l�uft und letztlich �fters stehen bleibt. Der Unterschied zwischen beiden Agenten ist besonders hoch in Szenarien mit gr��erem Anteil an Hindernissen.
Ansonsten liegt der intelligente Agent wieder eindeutig vorne, beherrscht aber besonders gut Szenarien mit hohem ``Verkn�pfungsfaktor'' (\(1.0\)) der geringem Anteil an Hindernissen (\(0.1\)), bei denen er bis zu etwa 15\% �ber dem Ergebnis des einfachen Agenten liegt.\\
Dies liegt daran, dass Szenarien mit hohem ``Verkn�pfungsfaktors'' bedeuten, dass alle Hindernisse zusammenh�ngend einen gro�en Block bilden und somit dem Szenario ohne Hindernissen �hnlich sind, auf dem dieser Agent ja besonders gut abschneidet. In zerkl�ftete Szenarien hat der Algorithmus dagegen Schwierigkeiten um andere Agenten �berhaupt zu Gesicht bekommen, der Vorteil der Verteilung f�llt also zu einem Teil weg. 

Dies best�tigt auch ein Durchlauf bei dem Behinderungen der Sicht durch Hindernisse deaktiviert sind. Hierbei erreicht der intelligente Agent im Szenario (\(0.4\), \(0.1\)) statt TODO evtl weg

\section{Random Neighbor und One Direction Change}

Wesentlicher Punkt bei beiden Szenarien ist, dass der jetzige Ort des Zielobjekts maximal zwei Felder (die Standardgeschwindigkeit des Zielobjekts in den Tests) vom Ort in der vorangegangenen Zeiteinheit entfernt ist. Somit ist ein lokales Einfangen eher von Relevanz, wenn auch das Zielobjekt grunds�tzlich schneller als andere Agenten ist.\\

Dementsprechend ist der einfache Agent bei einem Hindernis-Anteil von \(0.0\) bis \(0.1\) besser als alle anderen Agenten und dementsprechend ist bei allen Tests der zuf�llige Agent weit abgeschlagen.
Ab einem Anteil von \(0.2\) liegt jedoch der intelligente Agent vorne. Dies liegt schlicht an der Zahl der Agenten relativ zur hindernisfreien Fl�che, da sich die Agenten in m�glichst gro�em Abstand zueinander positionieren.

Im Falle des ``One Direction Change'' bewegt sich der Agent im Grunde nur etwas schneller, da er es vermeidet, auf das urspr�ngliche Feld zur�ckzukehren.
TODO? Vielleicht sogar Random Neighbor raus...

\section{Intelligent Open}

\section{Intelligent Hide}

TODO:Beide gleiche Ergebnisse?Source pr�fen


\section{Always Same Direction}

TODO

\section{LCS}

Wird weiter unten besprochen.





\section{Zusammenfassung}

Wie wir gesehen haben gibt es also Szenarien in denen Abdeckung kaum eine Rolle spielt und lokale Entscheidungen eine wesentliche Rolle spielen. Dies wird es erleichtern, geeignete Szenarien im Kapitel ``Kommunikation'' zu finden.




TODO Anpassung LCS an unterschiedliche Sichtreichweiten?
