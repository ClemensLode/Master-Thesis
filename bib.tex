\begin{thebibliography}{99}
\bibitem{Butz} {\sc Butz, M. \& Wilson, S.W.:}  \textit{An Algorithmic Description of XCS}, 2001.
In P-L. Lanzi, W. Stolzmann \& S.W. Wilson (eds) Advances in Learning Classifier Systems: IWLCS 2000. Springer, pp253-272.

\bibitem{Wilson} {\sc Wilson, S.W.:} \textit{Classifier Fitness Based on Accuracy}, 1995
Evolutionary Computation 3(2): 149-175

\bibitem{Butz2000} {\sc Butz, M.:} \textit{XCSJava 1.0: An implementation of the XCS classifier system in Java}
IlliGAL Report No. 2000027, June, 2000
\url{http://www.illigal.uiuc.edu/pub/papers/IlliGALs/2000027.ps.Z}

\bibitem{Bull} {\sc Larry Bull:}  \textit{A Simple Accuracy-Based Learning Classifier System}, 
\url{http://www2.cmp.uea.ac.uk/~it/ycs/ycs.pdf}

\bibitem{Hamer} 
{\sc Carol Hamer}, \textit{J2ME Games With MIDP2}, Apress, 2004,
ISBN 1-590-59382-0
\url{http://www.java-tips.org/java-me-tips/midp/how-to-create-a-maze-game-in-j2me-3.html}

\bibitem{Butz2003} {\sc M. V. Butz, K. Sastry, and D. E. Goldberg:} \textit{�Tournament selection: Stable fitness pressure in XCS,�}
 in Lecture Notes in Computer Science,
E. Cant�-Paz, J. A. Foster, K. Deb, D. Davis, R. Roy, U.-M. O�Reilly,
H.-G. Beyer, R. Standish, G. Kendall, S. Wilson, M. Harman, J.
Wegener, D. Dasgupta, M. A. Potter, A. C. Schultz, K. Dowsland, N.
Jonoska, and J. Miller, Eds. Chicago, IL, Jul. 12�16, 2003, vol. 2724,
Proc. Genetic and Evol. Comput., pp. 1857�1869.

\bibitem{Butz2005} {\sc Martin V. Butz, David E. Goldberg, Pier Luca Lanzi}, \textit{Gradient descent methods in learning classifier systems: Improving XCS performance in multistep problems}
IEEE Transaction on Evolutionary Computation, 9(5):452�473, October 2005.

OCS, Central planning:
\bibitem{Takadama} {\sc Keiki Takadama, Koichiro Hajiri, Tatsuya Nomura, Michio Okada, Shinichi Nakasuka, Katsunori Shimohara} \textit{Learning model for adaptive behaviors as an organized group of swarm robots}
In Artif Life Robotics (1998) 2 : 123-128, ISAROB 1998



\bibitem{Butz2006} {\sc Martin V. Butz} \textit{The XCS Classifier System}
In Studies in Fuzziness and Soft Computing, Springer, 2006, pp51-64.
ISBN 978-3-540-25379-2

\bibitem{Butz2006a} {\sc Martin V. Butz} \textit{Simple Learning Classifier Systems}
In Studies in Fuzziness and Soft Computing, Springer, 2006, pp51-64.
ISBN 978-3-540-25379-2

\bibitem{Wilson1995} {\sc Stewart W. Wilson:} \textit{Classifier Fitness Based on Accuracy}. 
Evolutionary Computation 3(2): 149-175 (1995)


\bibitem{Banzhaf} {\sc W. Banzhaf, J. Daida, A. E. Eiben, M. H. Garzon, V. Honavar, M. Jakiela and R. E. Smith} \textit{�Extending the representation of classifier conditions, Part I: From binary to Messy coding,�}
in Proc. Genetic Evol. Comput. Conf., Eds., 1999b, pp. 337�344.

\bibitem{Barry} {\sc A. M. Barry} \textit{�The stability of long action chains in XCS,�}, 
Soft Comput.�A Fusion Foundations, Methodologies, Applicat., vol. 6, no.
3�4, pp. 183�199, 2002.

\bibitem{xcs1} \textit{�Classifier fitness based on accuracy,�} Evol. Comput., vol. 3,
no. 2, pp. 149�175, 1995

\bibitem{xcs2} {\sc J. R. Koza,W. Banzhaf, K. Chellapilla, K. Deb,
M. Dorigo, D. B. Fogel, M. H. Garzon, D. E. Goldberg, H. Iba, and R.
Riolo} \textit{�Generalization in the XCS classifier system,�}
in Proc. 3rd Ann. Conf. Genetic Program., , Eds., 1998, pp. 665�674.

\bibitem{xcslib} {\sc P. L. Lanzi} \textit{The XCS library}
\url{http://xcslib.sourceforge.net}

\bibitem{Lujan} {\sc Alejandro Lujan, Richard Werner, Azzedine Boukerche} \textit{Generation of Rule-based Adaptive Strategies for a
Collaborative Virtual Simulation Environment}
PARADISE Research Laboratory
University of Ottawa TODO
HAVE 2008 � IEEE International Workshop on
Haptic Audio Visual Environments and their Applications
Ottawa � Canada, 18-19 October 2008

\bibitem{Lanzi} {\sc Pier Luca Lanzi, Daniele Loiacono, Stewart W. Wilson, David E. Goldberg:} \textit{XCS with Computed Prediction in Continuous Multistep
Environments}
In Proceedings of the IEEE Congress on Evolutionary
Computation � CEC-2005, pages 2032�2039, Edinburgh, UK, September
2005. IEEE.

Generalization in reinforcement
learning: Safely approximating the value
function.

TODO

\bibitem{Miyazaki} {\sc K. Miyazaki, M. Yamamura, S. Kobayashi} \textit{On the
rationality of profit sharing in reinforcement learning}
In Proceedings of the 3rd International Conference on
Fuzzy Logic, Neural Nets and Soft Computing, pages
285�288, 1994.

\bibitem{Benouhiba} {\sc Toufik Benouhiba, Jean-Marc Nigro} \textit{An evidential cooperative multi-agent system}
Laboratoire ISTIT, CNRS FRE 2732, Universite� de Technologie de Troyes, Troyes, France

\bibitem{Hiroyashu} {\sc Hiroyasu Inoue, Keiki Takadama, Katsunori Shimohara} \textit{Exploring XCS in Multiagent Environments}

This research was conducted as part of Research on Human
Communication with funding from the National Institute
of Information and Communications Technology.

GECCO 05 June 25.29, 2005, Washington, DC, USA.
Copyright 2005 ACM 1-59593-097-3/05/0006
TODO

\end{thebibliography}
