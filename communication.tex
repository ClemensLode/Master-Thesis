\chapter{Verz�gertes SXCS}

\section{Verz�gerter Reward}

Der wesentliche Unterschied zur ersten XCS Variante SXCS ist, dass jeglicher ermittelter \emph{reward} Wert und der jeweils zugeh�rige Faktor lediglich erst einmal zusammen mit den jeweiligen \emph{actionSets} in einer Liste (\emph{historicActionSet} TODO Bezeichnung) gespeichert werden und in jedem Schritt immer nur die \emph{classifiers} des \emph{actionSets} des �ltesten Eintrags in der \emph{historicActionSet} Liste aktualisiert wird. Somit haben wir also eine zeitlich beliebig verz�gerbare Aktualisierungsfunktion, welche uns erlaubt, mehrere gleichzeitig stattgefundene (aber erst verz�gert eintreffende, wegen z.B. Kommunikationsschwierigkeiten) Ereignisse zusammen auszuwerten. Dies ist eine wesentliche Voraussetzung f�r Kommunikation zwischen den Agenten. TODO


Die Funktion \emph{calculateReward()} ist identisch mit der in Kapitel ~\ref{calculateRewardLCS:fig}besprochenen Funktion bei der SXCS Variante ohne verz�gerten Reward.

\begin{program}
  \begin{verbatim}
/**
 * Diese Funktion verarbeitet den �bergebenen Reward und gibt ihn an die
 * zugeh�rigen ActionSets weiter. Wesentlicher Unterschied zum LCS ohne 
 * Verz�gerung ist, dass maxPrediction erst bei der endg�ltigen 
 * Verarbeitung des historicActionSets ermittelt wird.
 *
 * @param reward Wahr wenn der Zielagent in Sicht war.
 * @param best_value Bester Wert des vorangegangenen actionSets
 * @param is_event Wahr wenn diese Funktion wegen eines Ereignisses, d.h.
 *        einem positiven Reward, aufgerufen wurde
 */

  public void collectReward(
                boolean reward, double best_value, boolean is_event) {
    double corrected_reward = reward ? 1.0 : 0.0;

  /**
   * Aktualisiere eine ganze Anzahl von Eintr�gen im historicActionSet
   */
     for(int i = 0; i < action_set_size; i++) {

  /**
   * Benutze aufsteigenden bzw. absteigenden Reward bei einem positiven 
   * bzw. negativen Ereignis
   */
       if(is_event) {
         corrected_reward = reward ? 
           calculateReward(i, action_set_size) : 
           calculateReward(action_set_size - i, action_set_size);
       }

  /**
   * F�ge den ermittelten Reward zum historicActionSet
   */
       historicActionSet.get(start_index - i).
         addReward(corrected_reward, factor);

    }

\end{verbatim}
  \caption{Zweites Kernst�ck des verz�gerten SXCS-Algorithmus (\emph{collectReward()} - Verteilung des Rewards auf die ActionSets)}
\end{program}

\begin{program}
  \begin{verbatim}

/**
 * Der erste Teil der Funktion ist identisch mit dem calculateNextMove
 * der LCS Variante ohne Kommunikation. Der Zusatz ist, dass beim 
 * �berlauf die im HistoricActionSet gespeicherte Rewards verarbeitet
 * werden
 */

  public void calculateNextMove(long gaTimestep) {
 
  // ... 

  /**
   * HistoryActionSet voll? Dann verarbeite den dort gespeicherten Reward
   */
     if (historicActionSet.size() > Configuration.getMaxStackSize()) {
      HistoryActionClassifierSet first = historicActionSet.pop();
      last.processReward(historicActionSet.getFirst().getBestValue());
    }
  }
\end{verbatim}
  \caption{Auszug aus dem dritten Kernst�ck des verz�gerten SXCS-Algorithmus (\emph{calculateNextMove()})}
\end{program}

\begin{program}
  \begin{verbatim}

/**
 * Zentrale Routine des HistoryActionSets zur Verarbeitung aller 
 * eingegangenen Rewards bis zu diesem Punkt.
 */

  public void processReward(double max_prediction) {

    double max = 0.0;
    double max_factor = 0.0;
  /**
   * Finde das gr��te reward / factor Paar TODO Verbessern
   */
    for(RewardHelper r : reward) {
      if(r.reward >= max && r.factor >= max_factor) {
        max = r.reward;
        max_factor = r.factor;
      }
    }
  /**
   * Aktualisiere den Reward mit den ermittelten Werten und dem
   * �bergebenen maxPrediction Wert
   */
    actionClassifierSet.updateReward(max, max_prediction, max_factor);
  }
\end{verbatim}
  \caption{Auszug aus dem vierten Kernst�ck des verz�gerten SXCS-Algorithmus (Verarbeitung des Rewards, \emph{processReward()})}
\end{program}


